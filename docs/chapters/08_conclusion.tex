% !TeX root = ../main.tex
% Add the above to each chapter to make compiling the PDF easier in some editors.
\chapter{Conclusion}\label{chapter:conclusion}
We successfully formalized the polynomial-commitment-scheme-typical security properties (polynomial binding, evaluation binding, and hiding) for the KZG, as well as the additional property of knowledge soundness that is commonly required for SNARK constructions \parencite{thalerbook,plonk,sonic,halo}. Furthermore, we formalized the proofs in a game-based format for additional rigourness and thus trust in the security of our proofs.
This work is hence a strong guarantee for the security of the KZG, proving it a secure cryptographic primitive.

\section*{Future Work}
There are three particular directions for future work: firstly, further development on the formalization of the KZG, secondly continuing from the KZG and formalizing more complex KZG- or polynomial-commitment-scheme-based cryptographic primitives, and thirdly, formalizing cryptographic primitives assumed by the KZG.

\subsection*{Further Development on the Formalization}
Isabelle supports code generation for several languages, including Haskell, OCaml and SML \parencite{code_gen} and, recently added, Go \parencite{go_codegen}. Hence, it would be desirable to extract a formally verified implementation out of this formalization, thus erasing the potential risk of implementation errors, which is a common root for security bugs in cryptographic implementations (see the recent Nova incident \parencite{nova_bug}, the 00-Plonk incident \parencite{00Plonk} or the \textit{Frozen Heart} incident \parencite{FHPlonk}). Due to the time-constraint setting of this thesis, we were not able to execute the code generation, but will do this for sure once the thesis is submitted.

Furthermore, a batched version that opens multiple commitments, to respectively multiple polynomials, for one value as proposed in \parencite{plonk} and \parencite{sonic} could be formalized. This would improve verifier time for SNARKs such as Plonk, Sonic, and Halo (that make use of this feature), should they be formalized and extracted via code generation, to obtain a formally verified SNARK.

\subsection*{Formalizing KZG-based Cryptography}
As pointed out in the introduction, SNARKs are a cryptographic primitive that is heavily used to secure funds in blockchain applications, however, to our knowledge, no SNARKs that rely on PCS's have been formally verified yet, although they are used in practice (see the introduction). This led to vulnerabilities being discovered only after they already have been implemented and used (see \parencite{FHBulletproofs}). Now, that we have formalized a PCS, namely the KZG, with this work, we open the possibility of formalizing PCS-based SNARKs e.g. Plonk, Sonic, Marlin, Nova, and Halo. Together with code generation, this would present the opportunity for a fully formally verified SNARK. 

\subsection*{Formalizing KZG primitives}
The KZG relies on pairings, which we have abstracted in the preliminaries and assumed in the locale to prove that the KZG is secure. Pairings are a cryptographic primitive, that is typically constructed using pairing-friendly elliptic curves such as the Barreto-Nährig\parencite{BN-EC} or the Baretto-Lynn-Scott\parencite{BLS-EC} elliptic curves \parencite{boneh_shoup}. However, to our knowledge, none of these have been formalized yet. To obtain a fully formally verified KZG implementation from the code generation, a pairing-friendly elliptic curve needs to be formally verified.