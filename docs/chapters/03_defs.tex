% !TeX root = ../main.tex
% Add the above to each chapter to make compiling the PDF easier in some editors.
\chapter{KZG Definition}\label{chapter:defs}
In this chapter, we give the concrete construction, as well as an intuition of the KZG.
Furthermore, we outline how the KZG has been formalized, note, however, that the definitions were formalized in a practical course and are not part of this thesis. 

\section{Intuition}
In this section, we want to give an intuitive idea of how the KZG constructs a PCS, specifically, how a commitment is constructed and how the createWitness function works. 
We neglect the exact setup process for the intuition but note that we obtain the ability to evaluate a polynomial on a fixed unknown random point $\alpha$.

The commitment $C$ is the evaluation at the unknown random point, intuitively this is sound because of the Schwarzt-Zippel lemma, which states that the probability of two different polynomials evaluating to the same value at a random point is negligible \parencite{thalerbook}.

To create a witness for a point to reveal, consider the following:
for any point $i$, any (non-trivial) polynomial $\phi(x)$ can be expanded to $\phi(x)=\psi(x)*(x-i) + \phi(i)$, for $\psi(x) = \frac{\phi(x)-\phi(i)}{(x-i)}$. Note that this equation is equivalent to $\phi(\alpha)=\psi(\alpha)*(\alpha-i) + \phi(i)$ (i.e. the equation evaluated at the unknown random point), except for negligible probability, due to the Schwartz-Zippel lemma. We use $\psi(\alpha)$ as the witness $\omega$, furthermore note that $\phi(\alpha)$ is the commitment $C$. 
This choice of the witness is sound, as the equation to be checked by the verifier,
$C=\omega*(\alpha-i) + \phi(i) \iff \phi(\alpha)=\psi(\alpha)*(\alpha-i) + \phi(i)$, holds if and only if the claimed $\phi(i)$ is exactly the evaluation of $\phi(x)$ at point i, as $\phi(x)=\psi(x)*(x-i) + \phi(i) \iff 
\phi(x)-\phi(i)=\psi(x)*(x-i)$ and the left-hand-side must be zero for $x=i$.

In the real protocol, the values for the unknown random point, the commitment and the witness are provided in a group (e.g. $\alpha$ is given as $\mathbf{g}^\alpha$) to hide them and a pairing is used to check the equation for the witness.

\section{Definition}
\label{Def}
The KZG is a polynomial commitment scheme as defined in definition \ref{PCS_def}. In this section, we outline the PCS-constructing functions based on the original paper \parencite{KZG}:
\begin{itemize}
    \item \textbf{KeyGen($\kappa,t$) $\rightarrow \mathbb{G}_p, \mathbb{G}_T, e, PK$} 
    
    takes a security parameter $\kappa$ and generates instances for the algebraic primitives the KZG needs, which are: 
    \begin{itemize}
        \item two cyclic groups, $\mathbb{G}_p$ and $\mathbb{G}_T$, of prime order $p$, where $p\ge2^{2^\kappa}$.
        \item a pairing function $e: \mathbb{G}_p \times \mathbb{G}_p \rightarrow \mathbb{G}_T$, as in \ref{pairings_def}, such that the t-SDH assumption holds.
    \end{itemize}
    \parencite{KZG}
    Additionally, KeyGen generates a toxic-waste secret key $\alpha$, from which it generates the public key $PK=(\mathbf{g}, \mathbf{g}^\alpha, \mathbf{g}^{\alpha^2},\dots,\mathbf{g}^{\alpha^t})\in \mathbb{G}_p^{t+1}$, for $t<2^\kappa$. KeyGen outputs the algebraic primitives and $PK$ to both the prover and verifier (as $pk$ and $ck$) \parencite{KZG}. 
    For good practice, the toxic waste secret key $\alpha$ should be deleted right after PK was generated, as $\alpha$ contains trap-door information. 
    \item \textbf{Commit(PK,$\phi(x)$) $\rightarrow C$}

    takes the public key $PK$ and a polynomial $\phi(x) \in \mathbb{Z}_p[X]$
    of maximum degree $t$, such that $\phi(x)=\sum_{0}^{deg(\phi)}\phi_jx^j$
    \parencite{KZG}. Commit returns the commitment $C=\mathbf{g}^{\phi(\alpha)}$ for $\phi$ as $C=\prod_{0}^{deg(\phi)}(\mathbf{g}^j)^{\phi_j}$\parencite{KZG}.
    The opening value for the KZG is simply the polynomial $\phi(x)$, which is why we omit to return it as well.

    \item \textbf{Verify($PK,C,\phi(x)$) $\rightarrow bool$}

    takes the public key $PK$, a commitment C and a polynomial $\phi(x) \in \mathbb{Z}_p[X]$ of maximum degree $t$, such that $\phi(x)=\sum_{0}^{deg(\phi)}\phi_jx^j$ \parencite{KZG}. Verify returns 1 if $C=\mathbf{g}^{\phi(\alpha)}$, otherwise 0.
    \item \textbf{CreateWitness($PK, \phi(x), i$) $\rightarrow i,\phi(i),\omega_i$ }

    takes the public key $PK$, a polynomial $\phi(x) \in \mathbb{Z}_p[X]$ of maximum degree $t$, such that $\phi(x)=\sum_{0}^{deg(\phi)}\phi_jx^j$, and a value $i\in\mathbb{Z}_p$ \parencite{KZG}. CreateWitness computes $\psi_i(x)=\sum_{0}^{deg(\psi)}\psi_jx^j$ as $\psi_i(x)=\frac{\phi(x)-\phi(i)}{(x-i)}$ and returns the tuple $(i,\phi(i),\mathbf{g}^{\psi(\alpha)})$, where $\mathbf{g}^{\psi(\alpha)}$ is computed, similar to the commit, as $\mathbf{g}^{\psi(\alpha)} = \prod_{0}^{deg(\psi)}(\mathbf{g}^j)^{\psi_j}$ \parencite{KZG}.
    \item \textbf{VerifyEval($PK,C,i.\phi(i),\omega_i$) $\rightarrow bool$}

    takes the public key $PK$, a commitment $C$, a claimed point $(i,\phi(i))$ and a witness $\omega_i$ for that point \parencite{KZG}. VerifyEval checks the equation $\phi(\alpha)=\psi(\alpha)(\alpha-i)+\phi(i)$ using the pairing $e$ as: $e(C,g) = e(\omega_i,\mathbf{g}^\alpha/\mathbf{g}^i)e(\mathbf{g},\mathbf{g})^{\phi(i)}$ and returns the result.
\end{itemize}





\section{Formalization}

In the formalization, we extract the algebraic primitives (i.e. the groups and the pairing) into a locale that can be instantiated. The major advantage of this is its compatibility with formalized instances of these algebraic primitives. Hence, the KZG's security could be instantiated for concrete constructions of pairings, such as the Barreto-Nährig\parencite{BN-EC} or Baretto-Lynn-Scott\parencite{BLS-EC} pairing friendly Elliptic Curves (ECs), should they be formalized.

We define \textit{KeyGen} as a wrapper around a function called \textit{Setup}, which mirrors the Setup function from \parencite{KZG} without generating the algebraic primitives, essentially \textit{Setup} is exposing the secret key $\alpha$ as well as the public key $PK$ and \textit{KeyGen} exposes only $PK$. This change is mentioned here, because we will often decompose \textit{KeyGen} into \textit{Setup}, to gain access to the secret key $\alpha$ in the proofs.

The remaining functions are defined trivially according to \ref{Def}, we omit the details as they are not part of this thesis and are irrelevant to the proofs.