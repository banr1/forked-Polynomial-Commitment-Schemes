% !TeX root = ../main.tex
% Add the above to each chapter to make compiling the PDF easier in some editors.
\chapter{Formalized Hardness Assumptions}\label{chapter:hardness_asmps_frml}
Before we continue with the formalization of the KZG's security in the next chapter, we formalize the hardness assumptions according to \ref{hardness_asmp_defs} in this chapter. We define them as games, such that we can target them in game-based proofs against an adversary.

\section*{Discrete Logarithm}
We formalize the DL game according to \ref{DL_def} as follows: 
\begin{lstlisting}[language=isabelle]
    TRY do {
        a  $\leftarrow$ sample_uniform p;
        a' $\leftarrow \mathcal{A}$ $\mathbf{g}^a$;
        return_spmf ($a=a'$)
    } ELSE return_spmf False
\end{lstlisting}
We sample $\text{a} \in \mathbb{Z}_p$ uniformly random using \textit{sample\_uniform} and supply the adversary $\mathcal{A}$ with $\mathbf{g}^a$. The adversary returns a guess for a, namely a'. The adversary wins the game if and only if a equals a'.

\section*{t-Strong Diffie Hellmann}
We formalize the t-SDH game according to \ref{tSDH_def} as follows: 
\begin{lstlisting}[language=isabelle]
    TRY do {
        $\alpha$  $\leftarrow$ sample_uniform p;
        let instc = map $\bigl(\lambda i. \mathbf{g}^{(\alpha^i)}\bigr)$ [0..<t+1];
        (c,g) $\leftarrow \mathcal{A}$ instc;
        return_spmf ($\text{g}=\mathbf{g}^{\frac{1}{\alpha+c}}$)
    } ELSE return_spmf False
\end{lstlisting}
We sample $\alpha\in\mathbb{Z}_p$ uniformly random using \textit{sample\_uniform} and compute the t-SDH instance $(\mathbf{g}, \mathbf{g}^{\alpha}, \mathbf{g}^{\alpha^2}, \dots, \mathbf{g}^{\alpha^t})$ as instc. The adversary $\mathcal{A}$ receives the t-SDH instance and returns a field element $\text{c}\in\mathbb{Z}_p$ and a group element $\text{g}\in\mathbb{G}_p$. The adversary wins if and only if the group element g is equal to $\mathbf{g}^{\frac{1}{\alpha+c}}$.

Note, that the t-SDH assumption was formalized in the practical course and thus is not part of this thesis. Nevertheless, we outlined it for completeness. 

\section*{t-Strong Bilinear Diffie Hellmann}
We formalize the t-BSDH game according to \ref{tBSDH_def} as follows: 
\begin{lstlisting}[language=isabelle]
    TRY do {
        $\alpha$  $\leftarrow$ sample_uniform p;
        let instc = map $\bigl(\lambda i. \mathbf{g}^{(\alpha^i)}\bigr)$ [0..<t+1];
        (c,g) $\leftarrow \mathcal{A}$ instc;
        return_spmf ($\text{g}=(\text{e } \mathbf{g}\ \mathbf{g})^{\frac{1}{\alpha+c}}$)
    } ELSE return_spmf False
\end{lstlisting}
The game is equivalent to the t-SDH game except for the equality check and the adversary's return types. 
While the t-SDH adversary returns a group element from $\mathbb{G}_p$, the t-BSDH adversary returns a group element of the parings e target group $\mathbb{G}_T$. Accordingly, the equality check is performed on the target group, specifically on the t-SDH value passed through the pairing e: e $\mathbf{g}$ $\mathbf{g}^{\frac{1}{\alpha+c}} = \text{(e}$ $\mathbf{g}$ $\mathbf{g})^{\frac{1}{\alpha+c}}$.

Lastly, note that every game in this chapter can trivially be transformed to do the equality check in an assert statement and return True after that, without changing the according game's spmf. This will be useful for the security proofs in the next chapter. Proofs for each game are to be found in the respective Isabelle theories.