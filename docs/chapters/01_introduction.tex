% !TeX root = ../main.tex
% Add the above to each chapter to make compiling the PDF easier in some editors.

\chapter{Introduction}\label{chapter:introduction}
In the past 15 to 20 years the field of cryptography has shifted its focus from messages (e.g.\ hiding messages, encrypting messages etc.) to computation (e.g. multi-party computation, computation over encrypted data and proving computation).

The first efficient \textit{multi-party computation} (MPC) protocol was proposed in 2011 by Damgard et al.\parencite{SPDZ}. The solution to computation over encrypted data, which is \textit{fully homomorphic encryption} (FHE), was proposed by Gentry in 2009 \parencite{fhe} and followed by many FHE protocols \parencite{CKKS,TFHE,BGV,BFV}.
Finally, Groth proposed the first efficient construction for succinctly proving computation in 2016: an efficient \textit{succinct non-interactive argument of knowledge} (SNARK) \parencite{Groth16}. This started a wave of research on SNARKs \parencite{thalerbook,Groth16,Bulletproofs,marlin,sonic,plonk,halo,nova,ProtoStar}.

This work is focused on a cryptographic primitive that is frequently used in these new cryptographic protocols, specifically SNARKs\parencite{plonk,sonic,marlin,halo} and MPC \parencite{PCS_MPC,BDOZ, VSS_MPC, KZG}, namely the polynomial commitment scheme (PCS) constructed by Kate, Zaverucha and Goldberg (KZG)\parencite{KZG}.

A PCS is a two-party cryptographic protocol between a committer and a verifier. It allows the committer to commit to an arbitrary polynomial in a way that is both binding (i.e.\ the commitment for a polynomial is unique) and hiding from the verifier (i.e.\ the verifier cannot guess the polynomial from the commitment). Furthermore, a PCS allows to reveal points of the polynomial, such that only the points are revealed, but not the polynomial itself. 
%Intuitively, this is important for SNARKs, as correct computation in SNARKs is typically expressed via identities on polynomials. Hence a PCS is used to ensure that the prover cannot craft a malicious polynomial just in time to mock a correct computation.

Although PCS are frequently used to construct cryptographic protocols such as SNARKs, we are not aware of a formalization of any polynomial commitment scheme. With this work, we set out to change this and give the first formalization of a polynomial commitment scheme, specifically focusing on the security proofs, thus paving the way for formalized security in modern cryptography, specifically SNARKs.

However, we want to motivate this work not purely with academic reasons, but want to highlight why it is absolutely necessary by examples of applications. One such example is the Ethereum blockchain, which recently (on the 13th of March, 2024) upgraded its protocol to support EIP-4844, which uses the KZG to store commitments to data blobs \parencite{EIP4844}. As the blockchain with the largest market capitalization, after Bitcoin, of >440 billion USD \parencite*{CoinMarketCap}, as of writing this, a security flaw in the KZG would probably result in losses of billions. Besides Ethereum, the KZG is heavily used in other blockchains (e.g. Aztec\parencite*{Aztec} and Mina\parencite*{Mina}), blockchain applications and infrastructure, such as SNARK-based rollups (e.g. Scroll\parencite*{Scroll}, zkSync\parencite*{zksync}, and Taiko\parencite*{Taiko}), data-availability networks (e.g. Celestia\parencite*{Celestia}) and co-processors (e.g. Axiom\parencite*{Axiom}). Therefore, any flaw in the security of the KZG would have a detrimental practical impact. 

With this work, we aim to not only lay the foundation for formalized security of modern cryptographic protocols (like SNARKs) but also to strengthen the trust in the security of the KZG and thus in systems and applications that are built on it.

\section*{Related Work}
PCS constructions fall into three major groups, namely pairing-based, hash-based or inner-product-argument-based (IPA-based). The KZG is a pairing-based PCS.

Hash-based PCS constructions, like e.g. FRI\parencite{FRI}, are post-quantum secure and used in transparent SNARKs such as scalable transparent non-interactive arguments of knowledge (STARKs) \parencite{STARK, FRI}. Note, a SNARK is considered transparent if no trusted setup is needed  (i.e.\ a STARK) \parencite{STARK}. Furthermore, note, that the setup required by the KZG is trusted as it generates the secret key, which is a trapdoor-information for the scheme. IPA-based SNARKs, like e.g Bulletproofs \parencite{Bulletproofs}, are also transparent and use range checks to construct a PCS. Further pairing-based constructions, like e.g. Dory \parencite*{Dory} improve time-complexities on some operations for a trade-off in another, compared to the KZG. 
Note, that there are many more constructions for PCSs, e.g.\ DARK\parencite*{DARK}, Virgo \parencite*{Virgo} and Hyrax\parencite*{Hyrax}. We point the reader to \parencite*{Dory} for an overview including these and more polynomial commitment schemes.

The most notable advantage of the KZG over all other PCS constructions, we are aware of, is: Firstly, the KZG is explicitly mentioned in many SNARKs \parencite{plonk,sonic,marlin,halo,nova} and, secondly, the KZG is the most efficient one with regard to group operations \parencite{Dory}. 
Furthermore, the KZG is the first construction of a polynomial commitment scheme. 

We formalize our proofs for the KZG in CryptHOL\parencite{CryptHOL-AFP}, which is a framework to formalize game-based proofs in Isabelle/HOL.
Besides CryptHOL there are many frameworks and tools specifically for formal verification of cryptography, to mention a few: `A Framework for Game-Based Security Proofs` \parencite{game_based_coq} or CertiCrypt \parencite{crypto_coq} in Coq\parencite{Coq}, EasyCrypt\parencite{EasyCrypt}, CryptoVerify\parencite{CryptoVerif} and cryptoline \parencite{Cryptoline}. Furthermore, the interactive theorem prover Lean\parencite{Lean} has recently been used for formal verification of cryptographic protocols due to its extensive math library \textit{mathlib} \parencite{Lean_groth16}.
These tools and Isabelle, specifically CryptHOL, have been used to formally verify many cryptographic primitives and protocols, of which we will only highlight the ones that are closely related to our formalization.

Butler et al.\ use CryptHOL to formalize a general framework for commitment schemes from $\Sigma$-protocols in \parencite{sigma_commit_crypto}. Though their framework provides a good orientation for the proofs, it does not sufficiently capture the complexity of a polynomial commitment scheme.

Firsov and Unruh formalize security properties for zero-knowledge protocols from sigma-protocols in EasyCrypt \parencite{zk_easycrypt}.
Bailey and Miller formalize specific SNARK constructions that do not rely on commitment schemes or other similarly complex cryptographic primitives in Lean \parencite{Lean_groth16}. 

With our work, we want to lay the foundation to formalize modern, more complex, SNARKs (that rely on polynomial commitment schemes), prominent examples that explicitly mention the KZG include Plonk \parencite{plonk}, Halo\parencite{halo}, Sonic \parencite{sonic}, Marlin \parencite{marlin}, and Nova \parencite{nova}.

Bosshard, Bootle and Sprenger formally verify the sum-check protocol \parencite{sumcheck_Isabelle}, which is another cryptographic primitive that is used in SNARKs with fast provers, such as Lasso and Jolt \parencite{Lasso, jolt}. Though sum-checks are not directly related to SNARKs that use polynomial commitment schemes, their formalization, similarly to our formalization, forms a step towards formally verified complex SNARK proving systems and thus is worth acknowledging.

\section*{Contributions}

We build on our work in the practical course, where we showed correctness and \textit{polynomial binding} for the KZG, and complete the formalization of the common security properties for commitment schemes. We show \textit{hiding} and \textit{binding}, where binding is split up into the two properties \textit{polynomial-binding} (showed in the practical course) and \textit{evaluation-binding}. Additionally, we also formalize the security property \textit{knowledge-soundness}, which is required for some SNARK constructions \parencite*{sonic,plonk}. Hence, we formalize these three security properties  for the KZG:
\begin{itemize}
    \item \textbf{evaluation binding:}
    No efficient adversary can compute a commitment, witnesses and two values, $\phi(x)$ and $\phi(x)'$, that are accepted by the verifier for an arbitrary but equal point $i$, except for negligible probability.
    \item \textbf{hiding:}
    No efficient adversary can compute a degree $t$ polynomial from a commitment to that polynomial and t additional evaluations of the polynomial, except for negligible probability.
    \item \textbf{knowledge soundness:}
    Intuitively this property states, that an efficient adversary must know a polynomial $\phi$ and compute the commitment $C$ for $\phi$ honestly, to reveal points, except for negligible probability. 
\end{itemize} 
Additionally, we formalize the \textbf{batched version} of the KZG\parencite*{KZG}. The batched version is an extension of the KZG for two more functions, which allow to evaluate a degree $t$ polynomial at up to t points with one witness and one pairing check. We formalize the correctness of the batch version as well as the security properties of \textit{evaluation binding} and \textit{knowledge soundness}. Furthermore, we discuss \textit{hiding}, which we could not further investigate due to time constraints. For now, it remains an open question whether \textit{hiding} can be proved for the batched version. Note, that the property of polynomial binding is not changed by adding the two functions for the batch version and thus is already shown for the batched version. 

The original paper proofs are outlined in a reduction proof style, which is known to be error-prone \parencite{gamesB&R}. To enhance the security expression of the proofs we transform them into game-based proofs, which are particularly rigorous \parencite{shoup_games, gamesB&R}, before we formalize them. We use CryptHOL\parencite*{CryptHOL-AFP}, a framework specifically for game-based proofs in Isabelle, for our formalization. 
Additionally, we extend the theories $SPMF$ and $CryptHOL$ with some lemmas and definitions helpful for our formalization.

The complete project can be found on \href{https://github.com/tobias-rothmann/KZG-Polynomial-Commitment-Scheme}{github}\footnote{https://github.com/tobias-rothmann/KZG-Polynomial-Commitment-Scheme}.