% !TeX root = ../main.tex
% Add the above to each chapter to make compiling the PDF easier in some editors.
\chapter{Batch Version Definition}\label{chapter:batch_def}

In this chapter, we outline the batched version of the KZG, as defined in \parencite{KZG}, and show how we formalized it.
The batched version allows to verifiably open a commitment to a polynomial for up to $t$ points using only one witness \parencite{KZG}, note, that this batch version is different from the one mentioned in the Sonic\parencite{sonic} and Plonk\parencite{plonk} SNARK, where multiple commitments can be batch opened for one point. The \parencite{KZG} batched version is an extension of the KZG as defined in chapter \ref{chapter:defs} for the following two functions \parencite{KZG}: 
\begin{enumerate}
    \item \textbf{CreateWitnessBatch($PK,\phi(x),B$) $\rightarrow B,r(x),\omega_B$}
    
    takes the public key $PK$, a polynomial $\phi(x) \in \mathbb{Z}_p[X]$ of maximum degree $t$, such that $\phi(x)=\sum_{0}^{deg(\phi)}\phi_jx^j$, and a set $B\subset\mathbb{Z}_p$ of maximum $t$ values \parencite{KZG}. \textit{CreateWitnessBatch} computes $\psi_B(x)=\sum_{0}^{deg(\psi)}\psi_jx^j$ and $r(x)=\sum_{0}^{deg(r)}r_jx^j$ as $\psi_B(x)=\frac{\phi(x)-r(x)}{\prod_{i\in B}^{}(x-i)}$ and $r(x)= \phi(x) \text{ mod } \prod_{i\in B}^{}(x-i)$ and returns the tuple $(B,r(x),\mathbf{g}^{\psi(\alpha)})$, where $\mathbf{g}^{\psi(\alpha)}$ is computed, similar to the commit, as $\mathbf{g}^{\psi(\alpha)} = \prod_{0}^{deg(\psi)}(\mathbf{g}^j)^{\psi_j}$ \parencite{KZG}. Furthermore, note $\phi(x)=\psi_B(x)*(\prod_{i\in B}^{}(x-i)) + r(x)$ and thus $\forall i\in B.\ r(i)=\phi(i)$.

    \item \textbf{VerifyEvalBatch($PK,C,B,r(x),\omega_B$) $\rightarrow bool$}
    
    takes the public key $PK$, a commitment $C$, a set of positions $B$, a polynomial $r(x)$ which evaluates to the claimed evaluations of $\phi(x)$ at the positions in $B$, and a witness $\omega_B$ for the claimed points \parencite{KZG}. \textit{VerifyEvalBatch} checks the equation $\phi(x)=\psi_B(x)*(\prod_{i\in B}^{}(x-i)) + r(x)$ using the pairing $e$ as: $e(C,\mathbf{g}) = e(\mathbf{g}^{\prod_{i\in B}^{}(\alpha-i)}, \omega_B)e(\mathbf{g},\mathbf{g}^{r(\alpha)})$ and returns the result.


\end{enumerate}
Intuitively, \textit{CreateWitnessBatch} uses the same technique as \textit{CreateWitness}, but divides $\phi$ for multiple points, i.e. $\prod_{i\in B}^{}(x-i)$ instead of $(x-i)$, and respectively returns a polynomial with all evaluations for the points in the set $B$ instead of a value for a single point $i$, i.e. $r(x)$ for $B$, where $\forall i\in B.\ r(i)=\phi(i)$, instead of $\phi(i)$ for $i$.

Similarly, \textit{VerifyEvalBatch} checks an equation that matches the equation checked by \textit{CreateWitness}, except for the evaluation point $\phi(i)$, which is replaced by the evaluation polynomial $r(x)$, and the division for multiple positions $\prod_{i\in B}^{}(x-i)$ instead of one $(x-i)$, i.e. $\phi(x)=\psi_B(x)*(\prod_{i\in B}^{}(x-i)) + r(x)$ instead of $\phi(x)=\psi_i(x)*(x-i) + \phi(i)$.

\section*{Formalization}
We formalize the batch version as a locale extension of the KZG definition locale \textit{KZG\_Def}: 
\begin{lstlisting}[language=isabelle]
    locale KZG_BatchOpening_def = KZG_Def 
\end{lstlisting}

We formalize \textit{CreateWitnessBatch} as follows: 
\begin{lstlisting}[language=isabelle]
    definition CreateWitnessBatch :: "'a pk  $\Rightarrow$ 'e polynomial $\Rightarrow$ 'e eval_position set 
        $\Rightarrow$ ('e eval_position set $\times$ 'e polynomial $\times$ 'a eval_witness)"
    where
    "CreateWitnessBatch PK $\phi$ B =( 
      let r = r B $\phi$;
          $\psi$ = $\psi_B$ B $\psi$; 
          w_i = g_pow_PK_Prod PK $\psi$
      in
      (B, r, w_i)
    )"     
\end{lstlisting}
where $r$ is the function that computes the remainder of $\phi$ divided by $\prod_{i\in B}^{}(x-i)$, for a polynomial $\phi$ and a set of positions $B$, formally we define r in Isabelle as follows: 
\begin{lstlisting}[language=isabelle]
    fun r :: "'e eval_position set $\Rightarrow$ 'e polynomial $\Rightarrow$ 'e polynomial" where
        "r B $\phi$ = do {
        let prod_B = prod ($\lambda$i. [:-i,1:])  B;
        $\phi$ mod prod_B}"    
\end{lstlisting},
$\psi_B$ is the function that computes $\frac{\phi(x)-r(x)}{\prod_{i\in B}^{}(x-i)}$, formally we define $\psi_B$ in Isabelle as follows: 
\begin{lstlisting}[language=isabelle]
    fun $\psi_B$ :: "'e eval_position set $\Rightarrow$ 'e polynomial $\Rightarrow$ 'e polynomial" where
    "$\psi_B$ B $\phi$ = do {
        let prod_B = prod ($\lambda$i. [:-i,1:]) B;
        ($\phi$ - (r B $\phi$)) div prod_B}"
\end{lstlisting},
and $g\_pow\_PK\_Prod$ computes the $\mathbf{g}^{\phi(\alpha)}$ from the public key $PK = (\mathbf{g}, \mathbf{g}^\alpha, \mathbf{g}^{\alpha^2}, \dots, \mathbf{g}^{\alpha^t})$ and a polynomial $\phi$.

We formalize \textit{VerifyEvalBatch} as follows: 
\begin{lstlisting}[language=isabelle]
    definition VerifyEvalBatch :: "'a pk $\Rightarrow$ 'a commit $\Rightarrow$ 'e eval_position set 
        $\Rightarrow$ 'e polynomial $\Rightarrow$ 'a eval_witness $\Rightarrow$ bool"
    where 
        "VerifyEvalBatch PK C B r_x w_B = (
        let g_pow_prod_B = g_pow_PK_Prod PK (prod ($\lambda$i. [:-i,1:]) B);
            g_pow_r = g_pow_PK_Prod PK r_x in
        (e g_pow_prod_B w_B) $\oplus$ (e $\mathbf{g}$ g_pow_r) = e C $\mathbf{g}$))"     
\end{lstlisting}
where $e$ is a pairing function as defined in chapter \ref{chapter:defs} (KZG Def), $\oplus$ is the group operation in $\mathbb{G}_T$ and $g\_pow\_PK\_Prod$ is the group generator $\mathbf{g}$, exponentiated with the evaluation of a polnomial on $\alpha$. Hence, $g\_pow\_prod\_B$ and $g\_pow\_r$ are the polynomials $\prod_{i\in B}^{}(x-i)$ and $r(x)$ evaluated at alpha and the equation outlined in the definition of VerifyEvalBatch can be checked on the, to the commiter and verifier unknown, secret key $\alpha$.